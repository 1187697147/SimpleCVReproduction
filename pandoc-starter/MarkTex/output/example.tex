\documentclass{article}
\usepackage{lastpage}
\usepackage{ragged2e}

\usepackage{amsmath}%提供数学公式支持

\usepackage{graphics}%用于添加图片
\usepackage{graphicx}%加强插图命令
\newcommand{\figpath}[1]{contents/fig/#1}

\usepackage{fontspec}%用于配置字体
\usepackage[table]{xcolor}%用于各种颜色环境
\usepackage{enumitem}%用于定制list和enum
\usepackage{float}%用于控制Float环境,添加H参数(强制放在Here)
\usepackage[colorlinks,linkcolor=airforceblue,urlcolor=blue,anchorcolor=blue,citecolor=green]{hyperref}%用于超链接,另外添加该包目录会自动添加引用。

\usepackage[most]{tcolorbox}%用于添加各种边框支持
\usepackage[cache=true,outputdir=./out]{minted}%如果不保留临时文件就设置cache=false,如果输出设置了其他目录那么outputdir参数也有手动指定,否则会报错。
\tcbuselibrary{minted}%加载tcolorbox的代码风格

\usepackage[a4paper,left=3cm,right=3cm,top=3cm,bottom=3cm]{geometry}%用于控制版式
\usepackage{appendix}%用于控制附加文件
\usepackage{ifthen}

\usepackage{pdfpages}%用于支持插入其他pdf页
\usepackage{booktabs}%目前用于给表格添加 \toprule \midrule 等命令
\usepackage{marginnote} %用于边注
\usepackage[pagestyles,toctitles]{titlesec} %用于标题格式DIY
% \usepackage{fancyhdr}%用于排版页眉页脚
\usepackage{ragged2e} % 用于对齐
\usepackage{fixltx2e} %用于文本环境的下标
\usepackage{ulem} %用于好看的下划线、波浪线等修饰
\usepackage{pifont} %数学符号
\usepackage{amssymb} %数学符号

\usepackage{fontspec}
\setmainfont{DejaVu Serif}


\definecolor{langback}{RGB}{245,244,250}
\definecolor{langbacktitle}{RGB}{235,233,245}
\definecolor{langtitle}{RGB}{177,177,177}
\definecolor{langno}{RGB}{202,202,202}
\tcbset{arc=1mm}
\renewcommand{\theFancyVerbLine}{\sffamily\textcolor{langno}{\scriptsize\oldstylenums{\arabic{FancyVerbLine}}}}%重定义行号的格式
\newtcblisting{langbox}[1][tex]{%参考自https://reishin.me/tmux/ 的代码框样式
    arc=1mm,
    colframe=langbacktitle,
    colbacktitle=langbacktitle,
    coltitle=langtitle,
    fonttitle=\bfseries\sffamily,
    lefttitle=1mm,toptitle=0.5mm,bottomtitle=0.5mm,
    title = Code,
    drop shadow,
    listing engine=minted,
    minted style=colorful,
    minted language=#1,
    minted options={fontsize=\small,breaklines,autogobble,linenos,numbersep=2mm,xleftmargin=1mm},
    colback=langback,listing only,
    bottomrule=0mm,leftrule=0mm,toprule=0mm,rightrule=0mm,
    enhanced,
    % overlay={\begin{tcbclipinterior}\fill[langback] (frame.south west)rectangle ([xshift=5mm]frame.north west);\end{tcbclipinterior}}
}

\definecolor{boxback}{RGB}{245,246,250}
\newtcolorbox{markquote}{
    colback=boxback,fonttitle=\sffamily\bfseries,arc=0pt,
    boxrule=0pt,bottomrule=-1pt,toprule=-1pt,leftrule=-1pt,rightrule=-1pt,
    drop shadow,enhanced
}

\usepackage[UTF8,fontset=windowsnew,heading=true]{ctex}
\ctexset{
	section = {
	number = 第\chinese{section}章,
	format = \zihao{3}\bfseries,
	},
	subsection = {
	number = \arabic{section}.\arabic{subsection},
	format = \Large\bfseries
	},
	subsubsection = {
	number = \arabic{section}.\arabic{subsection}.\arabic{subsubsection},
	format = \Large\bfseries,
	},
    paragraph = {
	format = \large\bfseries,
	},
    subparagraph = {
	format = \large\bfseries,
	},
}

\setlength{\parindent}{2em}%设置首行缩进
\linespread{1.3}%设置行距

\setlength{\parskip}{0.5em}%设置段间距
\setcounter{tocdepth}{4}%设置目录级数
\setcounter{secnumdepth}{3}


\newtcbox{\inlang}[1][red]{on line,
arc=0pt,outer arc=0pt,colback=#1!10!white,colframe=#1!50!black,
boxsep=0pt,left=1pt,right=1pt,top=2pt,bottom=2pt,
boxrule=0pt,bottomrule=-1pt,toprule=-1pt,leftrule=-1pt,rightrule=-1pt}

\newlength\tablewidth


\definecolor{tablelinegray}{RGB}{221,221,221}
\definecolor{tablerowgray}{RGB}{247,247,247}
\definecolor{tabletopgray}{RGB}{245,246,250}
\definecolor{airforceblue}{rgb}{0.36, 0.54, 0.66}



\begin{document}
\normalsize

\section{1. 前言}


在2月10日,Faster RCNN专栏由pprp同学起了个头,文章地址见这里:\href{https://mp.weixin.qq.com/s/jhsXSr8xX8YvBK4jIpgX-g}{【Faster R{-}CNN】1. 梳理Faster R{-}CNN的四个模块},本着对公众号的每个专栏负责任的态度,我将在接下来的时间里将整个Faster RCNN的原理以及代码(陈云大佬的:https://github.com/chenyuntc/simple{-}faster{-}rcnn{-}pytorch)按照我的理解讲清楚并结束这个专题。


\section{2. Faster RCNN整体结构}


Faster RCNN的背景,介绍这些都没必要再次讲解了,这里我们直接再来复习一下Faster RCNN的整体结构,如下图所示。


\begin{center}
\vspace{\baselineskip}\includegraphics[width=0.8\textwidth]{images/8f7504e265cd051233aa7f1dfaedb69d.png}\vspace{\baselineskip}
\end{center}




可以看到Faster RCNN大概可以分成绿色描述的 $4$ 个部分,即:


\begin{itemize}
\item
DataSet:代表数据集,典型的比如VOC和COCO。
\item
Extrator:特征提取器,也即是我们常说的Backbone网络,典型的有VGG和ResNet。
\item
RPN:全称Region Proposal Network,负责产生候选区域(\inlang{\small{rois}}),每张图大概给出2000个候选框。
\item
RoIHead:负责对\inlang{\small{rois}}进行分类和回归微调。
\end{itemize}



所以Faster RCNN的流程可以总结为:


\textbf{原始图像}{-}{-}{-}>\textbf{特征提取}{-}{-}{-}{-}{-}{-}>\textbf{RPN产生候选框}{-}{-}{-}{-}{-}{-}>\textbf{对候选框进行分类和回归微调}。


\section{3. 数据预处理及实现细节}


首先让我们进入到这个Pytorch的Faster RCNN工程:\inlang{\small{https://github.com/chenyuntc/simple{-}faster{-}rcnn{-}pytorch}}。数据预处理的相关细节都在\inlang{\small{data}}这个文件夹下面,我画了一个流程图总结了Faster RCNN的预处理,如下:




\begin{center}
\vspace{\baselineskip}\includegraphics[width=0.8\textwidth]{images/1a90c84ab17a5b267f02c064b3a99178.png}\vspace{\baselineskip}
\end{center}


接下来我们结合一下我的代码注释来理解一下,首先是\inlang{\small{data/dataset.py}}。


\begin{langbox}[Py]
# 去正则化,img维度为[[B,G,R],H,W],因为caffe预训练模型输入为BGR 0-255图片,pytorch预训练模型采用RGB 0-1图片
def inverse_normalize(img):
    if opt.caffe_pretrain:
        # [122.7717, 115.9465, 102.9801]reshape为[3,1,1]与img维度相同就可以相加了,
        # pytorch_normalize之前有减均值预处理,现在还原回去。
        img = img + (np.array([122.7717, 115.9465, 102.9801]).reshape(3, 1, 1))
        # 将BGR转换为RGB图片(python [::-1]为逆序输出)
        return img[::-1, :, :]
    # pytorch_normalze中标准化为减均值除以标准差,现在乘以标准差加上均值还原回去,转换为0-255
    return (img * 0.225 + 0.45).clip(min=0, max=1) * 255

# 采用pytorch预训练模型对图片预处理,函数输入的img为0-1
def pytorch_normalze(img):
    """
    https://github.com/pytorch/vision/issues/223
    return appr -1~1 RGB
    """
    # #transforms.Normalize使用如下公式进行归一化
    # channel=(channel-mean)/std,转换为[-1,1]
    normalize = tvtsf.Normalize(mean=[0.485, 0.456, 0.406],
                                std=[0.229, 0.224, 0.225])
    # nddarry->Tensor
    img = normalize(t.from_numpy(img))
    return img.numpy()

# 采用caffe预训练模型时对输入图像进行标准化,函数输入的img为0-1
def caffe_normalize(img):
    """
    return appr -125-125 BGR
    """
    # RGB-BGR
    img = img[[2, 1, 0], :, :]  # RGB-BGR
    img = img * 255
    # 转换为与img维度相同
    mean = np.array([122.7717, 115.9465, 102.9801]).reshape(3, 1, 1)
    # 减均值操作
    img = (img - mean).astype(np.float32, copy=True)
    return img

# 函数输入的img为0-255
def preprocess(img, min_size=600, max_size=1000):
    # 图片进行缩放,使得长边小于等于1000,短边小于等于600(至少有一个等于)。
    # 对相应的bounding boxes 也也进行同等尺度的缩放。
    C, H, W = img.shape
    scale1 = min_size / min(H, W)
    scale2 = max_size / max(H, W)
    # 选小的比例,这样长和宽都能放缩到规定的尺寸
    scale = min(scale1, scale2)
    img = img / 255.
    # resize到(H * scale, W * scale)大小,anti_aliasing为是否采用高斯滤波
    img = sktsf.resize(img, (C, H * scale, W * scale), mode='reflect',anti_aliasing=False)
    #调用pytorch_normalze或者caffe_normalze对图像进行正则化
    if opt.caffe_pretrain:
        normalize = caffe_normalize
    else:
        normalize = pytorch_normalze
    return normalize(img)


class Transform(object):

    def __init__(self, min_size=600, max_size=1000):
        self.min_size = min_size
        self.max_size = max_size

    def __call__(self, in_data):
        img, bbox, label = in_data
        _, H, W = img.shape
        # 图像等比例缩放
        img = preprocess(img, self.min_size, self.max_size)
        _, o_H, o_W = img.shape
        # 得出缩放比因子
        scale = o_H / H
        # bbox按照与原图等比例缩放
        bbox = util.resize_bbox(bbox, (H, W), (o_H, o_W))

        # 将图片进行随机水平翻转,没有进行垂直翻转
        img, params = util.random_flip(
            img, x_random=True, return_param=True)
        # 同样地将bbox进行与对应图片同样的水平翻转
        bbox = util.flip_bbox(
            bbox, (o_H, o_W), x_flip=params['x_flip'])

        return img, bbox, label, scale

# 训练集样本的生成
class Dataset:
    def __init__(self, opt):
        self.opt = opt
         # 实例化类
        self.db = VOCBboxDataset(opt.voc_data_dir)
        #实例化类
        self.tsf = Transform(opt.min_size, opt.max_size)
    # __ xxx__运行Dataset类时自动运行
    def __getitem__(self, idx):
        # 调用VOCBboxDataset中的get_example()从数据集存储路径中将img, bbox, label, difficult 一个个的获取出来
        ori_img, bbox, label, difficult = self.db.get_example(idx)
        # 调用前面的Transform函数将图片,label进行最小值最大值放缩归一化,
        # 重新调整bboxes的大小,然后随机反转,最后将数据集返回
        img, bbox, label, scale = self.tsf((ori_img, bbox, label))
        # TODO: check whose stride is negative to fix this instead copy all
        # some of the strides of a given numpy array are negative.
        return img.copy(), bbox.copy(), label.copy(), scale

    def __len__(self):
        return len(self.db)

# 测试集样本的生成
class TestDataset:
    def __init__(self, opt, split='test', use_difficult=True):
        self.opt = opt
        # 此处设置了use_difficult,
        self.db = VOCBboxDataset(opt.voc_data_dir, split=split, use_difficult=use_difficult)

    def __getitem__(self, idx):
        ori_img, bbox, label, difficult = self.db.get_example(idx)
        img = preprocess(ori_img)
        return img, ori_img.shape[1:], bbox, label, difficult

    def __len__(self):
        return len(self.db)
\end{langbox}





接下来是\inlang{\small{data/voc\_dataset.py}},注释如下:


\begin{langbox}[Python]
class VOCBboxDataset:
    
    def __init__(self, data_dir, split='trainval',
                 use_difficult=False, return_difficult=False,
                 ):

        # if split not in ['train', 'trainval', 'val']:
        #     if not (split == 'test' and year == '2007'):
        #         warnings.warn(
        #             'please pick split from \'train\', \'trainval\', \'val\''
        #             'for 2012 dataset. For 2007 dataset, you can pick \'test\''
        #             ' in addition to the above mentioned splits.'
        #         )
        # id_list_file为split.txt,split为'trainval'或者'test'
        id_list_file = os.path.join(
            data_dir, 'ImageSets/Main/{0}.txt'.format(split))
        # id_为每个样本文件名
        self.ids = [id_.strip() for id_ in open(id_list_file)]
        # 写到/VOC2007/的路径
        self.data_dir = data_dir
        self.use_difficult = use_difficult
        self.return_difficult = return_difficult
        # 20类
        self.label_names = VOC_BBOX_LABEL_NAMES

    # trainval.txt有5011个,test.txt有210个
    def __len__(self):
        return len(self.ids)

    def get_example(self, i):
        #读入xml标签文件
        id_ = self.ids[i]
        anno = ET.parse(
            os.path.join(self.data_dir, 'Annotations', id_ + '.xml'))
        bbox = list()
        label = list()
        difficult = list()
        #解析xml文件
        for obj in anno.findall('object'):
            # 标为difficult的目标在测试评估中一般会被忽略
            if not self.use_difficult and int(obj.find('difficult').text) == 1:
                continue
            #xml文件中包含object name和difficult(0或者1,0代表容易检测)
            difficult.append(int(obj.find('difficult').text))
            # bndbox(xmin,ymin,xmax,ymax),表示框左下角和右上角坐标
            bndbox_anno = obj.find('bndbox')
            # #让坐标基于(0,0)
            bbox.append([
                int(bndbox_anno.find(tag).text) - 1
                for tag in ('ymin', 'xmin', 'ymax', 'xmax')])
            # 框中object name
            name = obj.find('name').text.lower().strip()
            label.append(VOC_BBOX_LABEL_NAMES.index(name))
        # 所有object的bbox坐标存在列表里
        bbox = np.stack(bbox).astype(np.float32)
        # 所有object的label存在列表里
        label = np.stack(label).astype(np.int32)
        # PyTorch 不支持 np.bool,所以这里转换为uint8
        difficult = np.array(difficult, dtype=np.bool).astype(np.uint8)  

        # 根据图片编号在/JPEGImages/取图片
        img_file = os.path.join(self.data_dir, 'JPEGImages', id_ + '.jpg')
        # 如果color=True,则转换为RGB图
        img = read_image(img_file, color=True)

        # if self.return_difficult:
        #     return img, bbox, label, difficult
        return img, bbox, label, difficult

    # 一般如果想使用索引访问元素时,就可以在类中定义这个方法(__getitem__(self, key) )
    __getitem__ = get_example

# 类别和名字对应的列表
VOC_BBOX_LABEL_NAMES = (
    'aeroplane',
    'bicycle',
    'bird',
    'boat',
    'bottle',
    'bus',
    'car',
    'cat',
    'chair',
    'cow',
    'diningtable',
    'dog',
    'horse',
    'motorbike',
    'person',
    'pottedplant',
    'sheep',
    'sofa',
    'train',
    'tvmonitor')
\end{langbox}



再接下来是\inlang{\small{utils.py}}里面一些用到的相关函数的注释,只选了其中几个,并且有一些函数没有用到,全部放上来篇幅太多:


\begin{langbox}[Python]
def resize_bbox(bbox, in_size, out_size):
    # 根据图片resize的情况来缩放bbox
    bbox = bbox.copy()
    #  #获得与原图同样的缩放比
    y_scale = float(out_size[0]) / in_size[0]
    x_scale = float(out_size[1]) / in_size[1]
    # #按与原图同等比例缩放bbox
    bbox[:, 0] = y_scale * bbox[:, 0]
    bbox[:, 2] = y_scale * bbox[:, 2]
    bbox[:, 1] = x_scale * bbox[:, 1]
    bbox[:, 3] = x_scale * bbox[:, 3]
    return bbox


def flip_bbox(bbox, size, y_flip=False, x_flip=False):
    # 根据图片flip的情况来flip bbox
    H, W = size #缩放后图片的size
    bbox = bbox.copy()
    if y_flip:  #进行垂直翻转
        y_max = H - bbox[:, 0]
        y_min = H - bbox[:, 2]
        bbox[:, 0] = y_min
        bbox[:, 2] = y_max
    if x_flip: #进行水平翻转
        x_max = W - bbox[:, 1]
        x_min = W - bbox[:, 3] #计算水平翻转后左下角和右上角的坐标
        bbox[:, 1] = x_min
        bbox[:, 3] = x_max
    return bbox

def random_flip(img, y_random=False, x_random=False,
                return_param=False, copy=False):
    # 数据增强,随机翻转
    y_flip, x_flip = False, False
    if y_random:
        y_flip = random.choice([True, False])
    # 随机选择图片是否进行水平翻转
    if x_random:
        x_flip = random.choice([True, False])

    if y_flip:
        img = img[:, ::-1, :]
    if x_flip:
        # python [::-1]为逆序输出,这里指水平翻转
        img = img[:, :, ::-1]

    if copy:
        img = img.copy()

    if return_param:
        #返回img和x_flip(为了让bbox有同样的水平翻转操作)
        return img, {'y_flip': y_flip, 'x_flip': x_flip}
    else:
        return img


\end{langbox}



至此,我们就可以很好的理解数据预处理部分了,这部分也是最简单的,下一节我们开始搭建模型。带注释的Faster RCNN完整代码版本等我更新完这个专题我再放出来。


\section{4. 思考}


可以看到在Faster RCNN的代码中,数据预处理是相对简单的,没有大量的增强操作(相比于YOLOV3来说),如果结合更多的数据增强操作是否可以获得更好的精度呢?感觉值得尝试一下。


\section{5. 附录}


\begin{itemize}
\item
https://blog.csdn.net/qq\_32678471/article/details/84776144
\item
https://zhuanlan.zhihu.com/p/32404424
\end{itemize}





欢迎关注GiantPandaCV, 在这里你将看到独家的深度学习分享,坚持原创,每天分享我们学习到的新鲜知识。( • ̀ω•́ )✧


有对文章相关的问题,或者想要加入交流群,欢迎添加BBuf微信:


\begin{center}
\vspace{\baselineskip}\includegraphics[width=0.8\textwidth]{images/6d4d967a3f880edb3bb9b494c8772a75.png}\vspace{\baselineskip}
\end{center}


\end{document}