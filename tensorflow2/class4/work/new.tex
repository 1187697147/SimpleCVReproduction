% -*- coding: utf-8 -*-
\documentclass{article}
\usepackage{listings}
\usepackage{ctex}
\usepackage{graphicx}
\usepackage[a4paper, body={18cm,22cm}]{geometry}
\usepackage{amsmath,amssymb,amstext,wasysym,enumerate,graphicx}
\usepackage{float,abstract,booktabs,indentfirst,amsmath}
\usepackage{array}
\usepackage{booktabs} %调整表格线与上下内容的间隔
\usepackage{multirow}
\usepackage{url}
\usepackage{diagbox}
\renewcommand\arraystretch{1.4}
\usepackage{indentfirst}
\setlength{\parindent}{2em}

\usepackage{listings}
\usepackage{xcolor}
\lstset{
    numbers=left, 
    numberstyle= \tiny, 
    keywordstyle= \color{ blue!70},
    commentstyle= \color{red!50!green!50!blue!50}, 
    frame=shadowbox, % 阴影效果
    rulesepcolor= \color{ red!20!green!20!blue!20} ,
    escapeinside=``, % 英文分号中可写入中文
    xleftmargin=2em,xrightmargin=2em, aboveskip=1em,
    basicstyle=\footnotesize,
    framexleftmargin=2em
} 


\geometry{left=2.8cm,right=2.2cm,top=2.5cm,bottom=2.5cm}
%\geometry{left=3.18cm,right=3.18cm,top=2.54cm,bottom=2.54cm}

\graphicspath{{figures/}}

\title{\heiti 实验一 远程过程调用中间件及数据访问中间件 }

\begin{document}
\maketitle
%\tableofcontents

\begin{center}
	%\LARGE{{\textbf{\heiti 实验一 远程过程调用中间件及数据访问中间件}}}
	%\newline
	%\newline
	%\date{2019年4月2日}
	\begin{table}[H]
		\centering
		\begin{tabular}{p{3cm}p{4cm}<{\centering}p{3cm}p{4cm}<{\centering}}
			班\quad 级: & ****** & 学\qquad 号: & ***** \\ \cline{2-2} \cline{4-4}
			姓\quad 名: & ****   & 指导老师:    & ***** \\ \cline{2-2} \cline{4-4}
		\end{tabular}
	\end{table}
\end{center}

\section{实习目的}


\section{实习要求}

\section{实习过程}



\section{实习总结}



\end{document}

%%%%%%%%%%%%%%%%%%%%%%%%%%%%Library%%%%%%%%%%%%%%%%%%%%%%%%%%%%%%%%%%%%%%%

% 1. 脚注用法
LaTeX\footnote{Latex is Latex} is a good software

%2. 强调
\emph{center of percussion} %[Brody 1986], %\lipsum[5]

%3. 随便生成一段话
\lipsum[4]

%4. 列条目
\begin{itemize}
	\item the angular velocity of the bat,
	\item the velocity of the ball, and
	\item the position of impact along the bat.
\end{itemize}

%5. 表格用法
\begin{table}[h]
	\centering
	\begin{tabular}{c|cc}
		\hline
		年份 & \multicolumn{2}{c}{指标}\\
		\hline
		2017 & 0.9997 & 0.0555 \\
		2018 & 0.9994 & 0      \\
		2019 & 0.9993 & 0      \\
		\hline
	\end{tabular}
	\caption{NAME}\label{SIGN}
\end{table}

\begin{center}
	\begin{tabular}{c|cclcrcc}
		\hline
		Year & theta  & $S_1^-$ & $S_2^-$ & $S_3^-$ & $S_4^+$ & $S_5^+$ & $S_6^+$ \\%表格标题
		\hline
		2016 & 1      & 0       & 0       & 0.0001  & 0       & 0       & 0       \\
		2017 & 0.9997 & 0.0555  & 0       & 0.2889  & 0.1844  & 0.463   & 0       \\
		2018 & 0.9994 & 0       & 0       & 0.0012  & 0.3269  & 0.7154  & 0       \\
		2019 & 0.9993 & 0       & 0       & 0       & 0.4325  & 1.0473  & 0       \\
		2020 & 0.9991 & 0       & 0       & 0       & 0.5046  & 1.2022  & 0       \\
		2021 & 0.999  & 0       & 0       & 0       & 0.5466  & 1.2827  & 0       \\
		2022 & 0.9989 & 0.0017  & 0       & 0.3159  & 0.562   & 1.2995  & 0       \\
		2023 & 0.9989 & 0       & 0       & 0.0109  & 0.5533  & 1.2616  & 0       \\
		2024 & 0.9989 & 0       & 0       & 0       & 0.5232  & 1.1769  & 0       \\
		2025 & 0.9989 & 0       & 0       & 0.1009  & 0.4738  & 1.0521  & 0       \\
		2026 & 0.9991 & 0       & 0       & 0       & 0.4071  & 0.8929  & 0       \\
		2027 & 0.9992 & 0.0004  & 0       & 0.1195  & 0.3248  & 0.7042  & 0       \\
		2028 & 0.9994 & 0.0164  & 0       & 0.046   & 0.2287  & 0.4902  & 0       \\
		2029 & 0.9997 & 0       & 0       & 0.0609  & 0.12    & 0.2545  & 0       \\
		2030 & 1      & 0       & 0       & 0       & 0       & 0       & 0       \\
		\hline
	\end{tabular}
\end{center}

%6. 数学公式
\begin{equation}
	a^2 = a * a\label{aa}
\end{equation}

\[
	\begin{pmatrix}{*{20}c}
		{a_{11} } & {a_{12} } & {a_{13} } \\
		{a_{21} } & {a_{22} } & {a_{23} } \\
		{a_{31} } & {a_{32} } & {a_{33} } \\
	\end{pmatrix}
	= \frac{{Opposite}}{{Hypotenuse}}\cos ^{ - 1} \theta \arcsin \theta
\]

\[
	p_{j}=\begin{cases} 0,              & \text{if $j$ is odd}  \\
		r!\,(-1)^{j/2}, & \text{if $j$ is even}
	\end{cases}
\]


\[
	\arcsin \theta  =
	\mathop{{\int\!\!\!\!\!\int\!\!\!\!\!\int}\mkern-31.2mu
		\bigodot}\limits_\varphi
	{\mathop {\lim }\limits_{x \to \infty } \frac{{n!}}{{r!\left( {n - r}
					\right)!}}} \eqno (1)
\]

%7. 双图并行
\begin{figure}[h]
	% 一个2*2图片的排列
	\begin{minipage}[h]{0.5\linewidth}
		\centering
		\includegraphics[width=0.8\textwidth]{./figures/0.jpg}
		\caption{Figure example 2}
	\end{minipage}
	\begin{minipage}[h]{0.5\linewidth}
		\centering
		\includegraphics[width=0.8\textwidth]{./figures/0.jpg}
		\caption{Figure example 3}
	\end{minipage}
\end{figure}

%8. 单张图片部分
\begin{figure}[h]
	%\small
	\centering
	\includegraphics[width=12cm]{./figures/mcmthesis-aaa.eps}
	\caption{Figure example 1} \label{fig:aa}
\end{figure}

%%%%%%%%%%%%%%%%%%%%%%%%%%%%%%%%%%%%%%%%%%%%%%%%%%%%%%%%%%%%%%%%%%%%%%%%%%%%%
\begin{minipage}{0.5\linewidth}
	\begin{tabular}{|c|c|c|}
		\hline
		\multicolumn{2}{|c|}{\multirow{2}{*}{合并}}&测试\\
		\cline{3-3}
		\multicolumn{2}{|c|}{}& 0.9997  \\
		\hline
		2019 & 0.9993 & 0 \\
		\hline
	\end{tabular}
\end{minipage}

\begin{minipage}{0.5\linewidth}
	\begin{tabular}{c|ccc}
		\hline
		年份 & \multicolumn{3}{c}{指标}\\
		\hline
		\multirow{3}{*}{合并} & 2017 & 0.9997 & 0.0555 \\
		                      & 2018 & 0.9994 & 0      \\
		                      & 2019 & 0.9993 & 0      \\
		\hline
	\end{tabular}
\end{minipage}



\begin{table}[h]
	\centering
	\begin{Large}
		\begin{tabular}{p{4scm} p{8cm} < {\centering}}
			\hline
			院\qquad 系: & 信息工程学院   \\
			\hline
			团队名称:    & PlantBook Team \\
			\hline
			分\qquad 组: & 第0组1号       \\
			\hline
			日\qquad 期: & 2017年10月28日 \\
			\hline
			指导教师:    & 吱吱吱         \\
			\hline
		\end{tabular}
	\end{Large}
\end{table}


\ctexset{
	section={
	  format+=\heiti \raggedright,
	  name={,、},
	  number=\chinese{section},
	  beforeskip=1.0ex plus 0.2ex minus .2ex,
	  afterskip=1.0ex plus 0.2ex minus .2ex,
	  aftername=\hspace{0pt}
	 },
}

\begin{table}[h]
	\centering
	\begin{Large}
		\begin{tabular}{p{3cm} p{7cm}<{\centering}}
			院  \qquad  系: & *** \\ \cline{2-2}
		\end{tabular}
	\end{Large}
\end{table}
\thispagestyle{empty}
\newpage
\thispagestyle{empty}
\tableofcontents
\thispagestyle{empty}
\newpage
\setcounter{page}{1}