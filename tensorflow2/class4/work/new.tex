% -*- coding: utf-8 -*-
\documentclass{article}
\usepackage{listings}
\usepackage{ctex}
\usepackage{graphicx}
\usepackage[a4paper, body={18cm,22cm}]{geometry}
\usepackage{amsmath,amssymb,amstext,wasysym,enumerate,graphicx}
\usepackage{float,abstract,booktabs,indentfirst,amsmath}
\usepackage{array}
\usepackage{booktabs} %调整表格线与上下内容的间隔
\usepackage{multirow}
\usepackage{url}
\usepackage{diagbox}
\renewcommand\arraystretch{1.4}
\usepackage{indentfirst}
\setlength{\parindent}{2em}

\usepackage{listings}
\usepackage{xcolor}
\lstset{
    numbers=left, 
    numberstyle= \tiny, 
    keywordstyle= \color{ blue!70},
    commentstyle= \color{red!50!green!50!blue!50}, 
    frame=shadowbox, % 阴影效果
    rulesepcolor= \color{ red!20!green!20!blue!20} ,
    escapeinside=``, % 英文分号中可写入中文
    xleftmargin=2em,xrightmargin=2em, aboveskip=1em,
    basicstyle=\footnotesize,
    framexleftmargin=2em
} 


\geometry{left=2.8cm,right=2.2cm,top=2.5cm,bottom=2.5cm}
%\geometry{left=3.18cm,right=3.18cm,top=2.54cm,bottom=2.54cm}

\graphicspath{{figures/}}

\title{\heiti 作业四 \qquad 连体网络MNIST优化}

\begin{document}
\maketitle
%\tableofcontents

\begin{center}
	%\LARGE{{\textbf{\heiti 实验一 远程过程调用中间件及数据访问中间件}}}
	%\newline
	%\newline
	%\date{2019年4月2日}
	\begin{table}[H]
		\centering
		\begin{tabular}{p{3cm}p{4cm}<{\centering}p{3cm}p{4cm}<{\centering}}
			年\quad 级: & 2020	 & 学\qquad 号: & 2016012963 \\ \cline{2-2} \cline{4-4}
			姓\quad 名: & 董佩杰   & 指导老师:    & 牛新 \\ \cline{2-2} \cline{4-4}
		\end{tabular}
	\end{table}
\end{center}

\section{实验要求}

1. 构建平衡测试集:

(1)正例(同一数字对)、反例(不同数字对)样例比为1:1。

(2)正例中,10个数字类型各占1/10。反例中,不同数字对的所有组合共$C^2_{10}=45$种,要求比例也为相同,即反例中,45种组合每个组合比例为1/45。测试集正反例总数不少于9000个。

(3)写一个测试集打印脚本,打印出构建好的测试集中类型数量信息,例如下:

\begin{lstlisting}
Positive (0,0): 450
Positive (1,1): 450
…
Positive (9,9): 450
Pos Total:4500
Negtive (0,1): 100
Negtive (0,2): 100
…	
Negtive (8,9): 100	
Neg Total:4500	
Total: 9000
\end{lstlisting}

2. 训练好网络后,在如上定义的测试集上测试精度Accuracy>0.9

提交内容需要有:
代码,文档(运行截图,结果截图(包括PR曲线,测试集数量统计打印列表等))

注意:孪生网络的输出是距离特征距离$e_w$,训练目的是用给定loss调整样本对的距离$e_w$,使得两个同样数字组成的样本对(正样本对)特征距离$e_w$近!两个不同数字组成的样本对(负样本对)的距离$e_w$远!。分类,是特征距离$e_w$孪生网络训练好之后的网络应用,用来判断输入是正样本对,还是负样本对。分类时,采用一个距离阈值t作为正、负样本对,$t<e_w$时,判定输入是正样本对,$t>e_w$时,判定为负样本对。t的选取可以通过网络在test数据集上的roc或pr曲线获得,一般取曲线的拐点。


\section{实验过程}

\subsection{数据集构建及均衡采样}

数据集构建部分代码:

\begin{lstlisting}
(x_train, y_train), (x_test, y_test) = mnist.load_data()
x_train, x_test = x_train / 255.0, x_test / 255.0

x_train = x_train[..., np.newaxis]
x_test = x_test[..., np.newaxis]

train_ds = tf.data.Dataset.from_tensor_slices(
    (x_train, y_train)).shuffle(1000).batch(BATCH_SIZE)

test_ds = tf.data.Dataset.from_tensor_slices(
    (x_test, y_test)).batch(BATCH_SIZE)
\end{lstlisting}

均衡采样:

\begin{lstlisting}
def balanced_batch(batch_x, batch_y, num_cls=10):
  batch_x = np.array(batch_x)
  batch_y = np.array(batch_y)
  # batch_x MNIST样本 batch_y, MNIST标签 num_cls
  # (数字类型个数,10,为了让10个数字类型都充分采样正负样本对)
  batch_size = len(batch_y)

  pos_per_cls_e = round(batch_size/2/num_cls/2)  # bs最少40+
  pos_per_cls_e *= 2

  # 根据y进行排序
  index = np.array(batch_y).argsort()
  ys_1 = batch_y[index]

  num_class = []
  pos_samples = []
  neg_samples = set()

  cur_ind = 0

  for item in set(ys_1):
      num_class.append((ys_1 == item).sum())
      # 记录有多少个一样的
      num_pos = pos_per_cls_e

      while(num_pos > num_class[-1]):
          num_pos -= 2
      # 找一个恰好大于num_class个数的值
      # 作为选取的正样本的个数

      pos_samples.extend(np.random.choice(
          index[cur_ind:cur_ind+num_class[-1]],
          num_pos, replace=False).tolist())
      # 正样本

      neg_samples = neg_samples | (
          set(index[cur_ind:cur_ind+num_class[-1]])-set(list(pos_samples)))
      cur_ind += num_class[-1]

  neg_samples = list(neg_samples)

  x1_index = pos_samples[::2]
  x2_index = pos_samples[1:len(pos_samples)+1:2]

  x1_index.extend(neg_samples[::2])
  x2_index.extend(neg_samples[1:len(neg_samples)+1:2])

  p_index = np.random.permutation(len(x1_index))  # shuffle操作

  x1_index = np.array(x1_index)[p_index]
  x2_index = np.array(x2_index)[p_index]

  r_x1_batch = batch_x[x1_index]
  r_x2_batch = batch_x[x2_index]
  # 得到最终重排后的结果

  r_y_batch = np.array(batch_y[x1_index] !=
                       batch_y[x2_index], dtype=np.float32)

  return r_x1_batch, r_x2_batch, r_y_batch
\end{lstlisting}

\subsection{孪生网络结构}

这部分是基于demo部分改动,生成10维的embedding。具体代码如下:

\begin{lstlisting}
class TFLeNet(tf.keras.Model):
	def __init__(self, num_classes=10):
		super(TFLeNet, self).__init__(name='TFLeNet')
		self.num_classes = num_classes
		self.conv1 = tf.keras.layers.Conv2D(6, kernel_size=(
			5, 5), strides=(1, 1), activation='relu', padding='same')
		self.pool1 = tf.keras.layers.MaxPooling2D(2, strides=(2, 2))
		self.conv2 = tf.keras.layers.Conv2D(16, kernel_size=(
			5, 5), strides=(1, 1), activation='relu', padding='valid')
		self.pool2 = tf.keras.layers.MaxPooling2D(2, strides=(2, 2))
		self.flatten = tf.keras.layers.Flatten()
		self.dense1 = tf.keras.layers.Dense(120, activation='relu')
		self.dense2 = tf.keras.layers.Dense(84, activation='relu')
		self.dense3 = tf.keras.layers.Dense(num_classes, activation='softmax')

	def call(self, inputs):
		x = self.conv1(inputs)
		x = self.pool1(x)
		x = self.conv2(x)
		x = self.pool2(x)
		x = self.flatten(x)
		x = self.dense1(x)
		x = self.dense2(x)
		return self.dense3(x)
\end{lstlisting}

\subsection{参数选取以及训练过程}

\begin{lstlisting}
model = TFLeNet()

optimizer = tf.keras.optimizers.Adam(lr)

train_loss = tf.keras.metrics.Mean(name='train_loss')
train_accuracy = tf.keras.metrics.Accuracy(name='train_accuracy')

test_loss = tf.keras.metrics.Mean(name='test_loss')
test_accuracy = tf.keras.metrics.Accuracy(name='test_accuracy')

def train_epoch(images, labels):
  with tf.GradientTape() as tape:
    data1, data2, label = balanced_batch(images, labels)
    output1 = model(data1)
    output2 = model(data2)
    E = dist(output1, output2)
    loss = loss_object(label, E)
  gradients = tape.gradient(loss, model.trainable_variables)
  optimizer.apply_gradients(zip(gradients, model.trainable_variables))
  E = K.cast(E >= e_w, dtype='float64')
  train_loss(loss)
  train_accuracy(label, E)

def plot_PR(precision, recall, area, thresholds):
  from matplotlib.ticker import FuncFormatter
  plt.style.use('ggplot')
  plt.plot(recall, precision, 'b.-', label="PR-curve", markersize=1)
  plt.plot(recall[:-1], thresholds, 'r.-', label='threshold', markersize=1)
  plt.xlabel("Recall")
  plt.ylabel("Precision")
  plt.xlim(xmin=0, xmax=1.02)
  plt.ylim(ymin=0, ymax=1.02)
  plt.legend()
  plt.grid(True)
  plt.gca().yaxis.set_major_formatter(FuncFormatter(to_percent))
  plt.gca().xaxis.set_major_formatter(FuncFormatter(to_percent))
  plt.savefig('PR-curve_auc_%.3f.png' % area, dpi=200)
  plt.close()
def test_epoch(images, labels):
  data1, data2, label = balanced_batch(images, labels)
  output1 = model(data1)
  output2 = model(data2)
  E = dist(output1, output2)

precision, recall, _thresholds = precision_recall_curve(label, E)

area = metrics.auc(recall, precision)

plot_PR(precision, recall, area, _thresholds)

loss = loss_object(label, E)
E = K.cast(E >= e_w, dtype='float64')

test_loss(loss)
test_accuracy(label, E)


for epoch in range(EPOCHS):
  train_loss.reset_states()
  train_accuracy.reset_states()
  test_loss.reset_states()
  test_accuracy.reset_states()

  for images, labels in train_ds:
    train_epoch(images, labels)

  for images, labels in test_ds:
    test_epoch(images, labels)

  print('Epoch {}, Train Loss: {:.3f}, Train acc: {:.3f} Test Loss: {:.3f} Test acc: {:.3f}'.
    format(epoch + 1, train_loss.result(), train_accuracy.result(),
      test_loss.result(), test_accuracy.result()))

\end{lstlisting}

参数主要是通过PR曲线进行选取的,训练结束后的PR曲线如下图:

\begin{figure}[H]
	%\small
	\centering
	\includegraphics[width=10cm]{./PR.png}
	\caption{PR曲线} \label{fig:aa}
\end{figure}

红色曲线是recall对应的阈值,在Precision=Recall的时候对应的阈值大概在0.4左右,所以设置$e_w=0.4$

训练日志如下:

\begin{figure}[H]
	%\small
	\centering
	\includegraphics[width=10cm]{./log.png}
	\caption{训练日志} \label{fig:aa}
\end{figure}




\end{document}

%%%%%%%%%%%%%%%%%%%%%%%%%%%%Library%%%%%%%%%%%%%%%%%%%%%%%%%%%%%%%%%%%%%%%

% 1. 脚注用法
LaTeX\footnote{Latex is Latex} is a good software

%2. 强调
\emph{center of percussion} %[Brody 1986], %\lipsum[5]

%3. 随便生成一段话
\lipsum[4]

%4. 列条目
\begin{itemize}
	\item the angular velocity of the bat,
	\item the velocity of the ball, and
	\item the position of impact along the bat.
\end{itemize}

%5. 表格用法
\begin{table}[h]
	\centering
	\begin{tabular}{c|cc}
		\hline
		年份 & \multicolumn{2}{c}{指标}\\
		\hline
		2017 & 0.9997 & 0.0555 \\
		2018 & 0.9994 & 0      \\
		2019 & 0.9993 & 0      \\
		\hline
	\end{tabular}
	\caption{NAME}\label{SIGN}
\end{table}

\begin{center}
	\begin{tabular}{c|cclcrcc}
		\hline
		Year & theta  & $S_1^-$ & $S_2^-$ & $S_3^-$ & $S_4^+$ & $S_5^+$ & $S_6^+$ \\%表格标题
		\hline
		2016 & 1      & 0       & 0       & 0.0001  & 0       & 0       & 0       \\
		2017 & 0.9997 & 0.0555  & 0       & 0.2889  & 0.1844  & 0.463   & 0       \\
		2018 & 0.9994 & 0       & 0       & 0.0012  & 0.3269  & 0.7154  & 0       \\
		2019 & 0.9993 & 0       & 0       & 0       & 0.4325  & 1.0473  & 0       \\
		2020 & 0.9991 & 0       & 0       & 0       & 0.5046  & 1.2022  & 0       \\
		2021 & 0.999  & 0       & 0       & 0       & 0.5466  & 1.2827  & 0       \\
		2022 & 0.9989 & 0.0017  & 0       & 0.3159  & 0.562   & 1.2995  & 0       \\
		2023 & 0.9989 & 0       & 0       & 0.0109  & 0.5533  & 1.2616  & 0       \\
		2024 & 0.9989 & 0       & 0       & 0       & 0.5232  & 1.1769  & 0       \\
		2025 & 0.9989 & 0       & 0       & 0.1009  & 0.4738  & 1.0521  & 0       \\
		2026 & 0.9991 & 0       & 0       & 0       & 0.4071  & 0.8929  & 0       \\
		2027 & 0.9992 & 0.0004  & 0       & 0.1195  & 0.3248  & 0.7042  & 0       \\
		2028 & 0.9994 & 0.0164  & 0       & 0.046   & 0.2287  & 0.4902  & 0       \\
		2029 & 0.9997 & 0       & 0       & 0.0609  & 0.12    & 0.2545  & 0       \\
		2030 & 1      & 0       & 0       & 0       & 0       & 0       & 0       \\
		\hline
	\end{tabular}
\end{center}

%6. 数学公式
\begin{equation}
	a^2 = a * a\label{aa}
\end{equation}

\[
	\begin{pmatrix}{*{20}c}
		{a_{11} } & {a_{12} } & {a_{13} } \\
		{a_{21} } & {a_{22} } & {a_{23} } \\
		{a_{31} } & {a_{32} } & {a_{33} } \\
	\end{pmatrix}
	= \frac{{Opposite}}{{Hypotenuse}}\cos ^{ - 1} \theta \arcsin \theta
\]

\[
	p_{j}=\begin{cases} 0,              & \text{if $j$ is odd}  \\
		r!\,(-1)^{j/2}, & \text{if $j$ is even}
	\end{cases}
\]


\[
	\arcsin \theta  =
	\mathop{{\int\!\!\!\!\!\int\!\!\!\!\!\int}\mkern-31.2mu
		\bigodot}\limits_\varphi
	{\mathop {\lim }\limits_{x \to \infty } \frac{{n!}}{{r!\left( {n - r}
					\right)!}}} \eqno (1)
\]

%7. 双图并行
\begin{figure}[h]
	% 一个2*2图片的排列
	\begin{minipage}[h]{0.5\linewidth}
		\centering
		\includegraphics[width=0.8\textwidth]{./figures/0.jpg}
		\caption{Figure example 2}
	\end{minipage}
	\begin{minipage}[h]{0.5\linewidth}
		\centering
		\includegraphics[width=0.8\textwidth]{./figures/0.jpg}
		\caption{Figure example 3}
	\end{minipage}
\end{figure}

%8. 单张图片部分
\begin{figure}[h]
	%\small
	\centering
	\includegraphics[width=12cm]{./figures/mcmthesis-aaa.eps}
	\caption{Figure example 1} \label{fig:aa}
\end{figure}

%%%%%%%%%%%%%%%%%%%%%%%%%%%%%%%%%%%%%%%%%%%%%%%%%%%%%%%%%%%%%%%%%%%%%%%%%%%%%
\begin{minipage}{0.5\linewidth}
	\begin{tabular}{|c|c|c|}
		\hline
		\multicolumn{2}{|c|}{\multirow{2}{*}{合并}}&测试\\
		\cline{3-3}
		\multicolumn{2}{|c|}{}& 0.9997  \\
		\hline
		2019 & 0.9993 & 0 \\
		\hline
	\end{tabular}
\end{minipage}

\begin{minipage}{0.5\linewidth}
	\begin{tabular}{c|ccc}
		\hline
		年份 & \multicolumn{3}{c}{指标}\\
		\hline
		\multirow{3}{*}{合并} & 2017 & 0.9997 & 0.0555 \\
		                      & 2018 & 0.9994 & 0      \\
		                      & 2019 & 0.9993 & 0      \\
		\hline
	\end{tabular}
\end{minipage}



\begin{table}[h]
	\centering
	\begin{Large}
		\begin{tabular}{p{4scm} p{8cm} < {\centering}}
			\hline
			院\qquad 系: & 信息工程学院   \\
			\hline
			团队名称:    & PlantBook Team \\
			\hline
			分\qquad 组: & 第0组1号       \\
			\hline
			日\qquad 期: & 2017年10月28日 \\
			\hline
			指导教师:    & 吱吱吱         \\
			\hline
		\end{tabular}
	\end{Large}
\end{table}


\ctexset{
	section={
	  format+=\heiti \raggedright,
	  name={,、},
	  number=\chinese{section},
	  beforeskip=1.0ex plus 0.2ex minus .2ex,
	  afterskip=1.0ex plus 0.2ex minus .2ex,
	  aftername=\hspace{0pt}
	 },
}

\begin{table}[h]
	\centering
	\begin{Large}
		\begin{tabular}{p{3cm} p{7cm}<{\centering}}
			院  \qquad  系: & *** \\ \cline{2-2}
		\end{tabular}
	\end{Large}
\end{table}
\thispagestyle{empty}
\newpage
\thispagestyle{empty}
\tableofcontents
\thispagestyle{empty}
\newpage
\setcounter{page}{1}